\newpage

\section{\textbf{CƠ SỞ LÝ THUYẾT}}

\subsection{Công nghệ Backend}

\textbf{Node.js} là môi trường runtime JavaScript phía server được xây dựng trên V8 Engine của Chrome. Với kiến trúc non-blocking I/O và event-driven, Node.js xử lý hiệu quả hàng nghìn kết nối đồng thời mà không cần tạo thread mới cho mỗi request - đặc biệt phù hợp cho ứng dụng E-commerce với lượng truy cập cao.

\textbf{Express.js} là framework web tối giản cho Node.js, cung cấp:
\begin{itemize}
    \item Routing: Quản lý endpoints (/api/products, /api/orders, /api/auth)
    \item Middleware: Xác thực JWT, kiểm tra quyền admin, kiểm duyệt nội dung
    \item Error Handling: Xử lý lỗi tập trung, trả response phù hợp
\end{itemize}

\textbf{MongoDB} là hệ quản trị CSDL NoSQL lưu trữ dữ liệu dạng JSON/BSON linh hoạt. Ưu điểm cho E-commerce: schema linh hoạt cho sản phẩm đa dạng, embedded documents (specs, colors), text search, aggregation pipeline cho báo cáo. \textbf{Mongoose ODM} cung cấp schema-based modeling với validation, middleware hooks và population.

\subsection{Công nghệ Mobile}

\textbf{Flutter} là UI toolkit của Google cho phép xây dựng ứng dụng đa nền tảng (Android, iOS, Web, Desktop) từ một codebase duy nhất. Sử dụng ngôn ngữ Dart với Hot Reload, widget-based architecture và biên dịch native cho hiệu suất cao.

Thư viện chính: \textbf{provider} (state management), \textbf{dio} (HTTP client), \textbf{cached\_network\_image} (caching ảnh), \textbf{flutter\_secure\_storage} (lưu JWT), \textbf{url\_launcher} và \textbf{app\_links} (Deep Link MoMo).

\subsection{Công nghệ Tích hợp}

\textbf{MoMo Payment Gateway:} Cổng thanh toán hàng đầu Việt Nam với 30+ triệu người dùng. Quy trình: App tạo request → Backend gọi MoMo API (HMAC-SHA256 signature) → Trả Deep Link → Mở MoMo app → User xác nhận → IPN callback → Cập nhật đơn hàng.

\textbf{Cloudinary:} CDN quản lý hình ảnh với upload API, tối ưu hóa tự động và phân phối toàn cầu.

\textbf{Bảo mật:} JWT authentication (Access Token 7 ngày + Refresh Token 30 ngày), bcrypt mã hóa mật khẩu, PIN Lock tùy chọn, HTTPS, input validation với express-validator.

\textbf{Voice Search (Speech-to-Text):} Tìm kiếm bằng giọng nói sử dụng mô hình GPT-4o-mini-transcribe:
\begin{itemize}
    \item \textbf{Ghi âm}: Package \texttt{record} với format WAV 16kHz mono
    \item \textbf{Transcribe API}: Gửi file audio đến server → Nhận text tiếng Việt
    \item \textbf{Prompt hints}: Tối ưu nhận dạng tên sản phẩm (iPhone, Samsung, Xiaomi...)
    \item \textbf{Quy trình}: User nhấn mic → Ghi âm → Stop → Upload → Transcribe → Tìm kiếm
\end{itemize}

\subsection{ABSA - Aspect-Based Sentiment Analysis}

\subsubsection*{Các nghiên cứu liên quan}
Các phương pháp truyền thống dựa trên TF-IDF + SVM/Naive Bayes gặp hạn chế trong nắm bắt ngữ nghĩa sâu. Sự phát triển của học sâu mở ra hướng tiếp cận mới:
\begin{itemize}
    \item \textbf{CNN}: Khai thác cụm từ quan trọng qua bộ lọc trượt, phù hợp văn bản ngắn
    \item \textbf{LSTM}: Giải quyết phụ thuộc dài hạn, phù hợp dữ liệu hội thoại
    \item \textbf{Transformer}: Cơ chế attention cho phép mô hình hóa quan hệ toàn cục giữa các token
    \item \textbf{BERT tiếng Việt}: PhoBERT, VisoBERT vượt trội so với mBERT đa ngôn ngữ
\end{itemize}

\subsubsection*{Định nghĩa bài toán}
ABSA là kỹ thuật phân tích cảm xúc nâng cao, chia thành 2 nhiệm vụ riêng biệt:

\begin{itemize}
    \item \textbf{Aspect Detection (AD)}: Xác định bình luận đang nói về khía cạnh nào của sản phẩm (Pin, Màn hình, Giao hàng...). Đây là bài toán phân loại đa nhãn (multi-label classification).
    \item \textbf{Sentiment Classification (SC)}: Với mỗi khía cạnh tìm được, xác định cảm xúc là Positive, Negative hay Neutral. Đây là bài toán phân loại đa lớp (multi-class classification).
\end{itemize}

\textbf{Ví dụ minh họa:}

\textit{Input:} "Pin dùng cả ngày không hết, nhưng camera hơi tệ trong điều kiện thiếu sáng."

\textit{Output:}
\begin{itemize}
    \item Battery → Positive (confidence: 0.95)
    \item Camera → Negative (confidence: 0.87)
\end{itemize}

\subsubsection*{Các khía cạnh (Aspects) được phân tích}

Hệ thống được huấn luyện nhận diện 11 khía cạnh đặc thù cho thương mại điện tử ngành điện thoại:

\begin{table}[H]
\centering
\caption{11 Aspects trong mô hình ABSA}
\footnotesize
\renewcommand{\arraystretch}{1.1}
\begin{tabular}{|c|l|p{7cm}|}
\hline
\textbf{STT} & \textbf{Aspect} & \textbf{Mô tả} \\
\hline
1 & Battery & Thời lượng pin, tốc độ sạc, công nghệ sạc nhanh \\
\hline
2 & Camera & Chất lượng ảnh, video, chụp đêm, selfie \\
\hline
3 & Performance & Tốc độ xử lý, đa nhiệm, gaming, chip \\
\hline
4 & Display & Độ sáng, màu sắc, tần số quét, độ phân giải \\
\hline
5 & Design & Ngoại hình, chất liệu, trọng lượng, màu sắc máy \\
\hline
6 & Packaging & Hộp đựng, phụ kiện đi kèm, seal nguyên vẹn \\
\hline
7 & Price & Giá cả, khuyến mãi, so với đối thủ \\
\hline
8 & Shop\_Service & Tư vấn, hỗ trợ, thái độ nhân viên \\
\hline
9 & Shipping & Tốc độ giao hàng, đóng gói vận chuyển \\
\hline
10 & General & Nhận xét tổng quan về sản phẩm \\
\hline
11 & Others & Các vấn đề khác không thuộc 10 aspects trên \\
\hline
\end{tabular}
\end{table}

\text{Lưu ý:} Packaging và Shipping là 2 khía cạnh quan trọng thường bị thiếu trong các bộ dữ liệu ABSA truyền thống, nhưng rất phổ biến trong bình luận mua hàng online.

\newpage
\subsubsection*{Kiến trúc mô hình (Model Architecture)}

\textbf{1. Core Model - VisoBERT}

Đồ án sử dụng VisoBERT làm backbone thay vì PhoBERT vì các lý do:
\begin{itemize}
    \item Huấn luyện trên 14GB dữ liệu mạng xã hội Việt Nam
    \item Hiểu tốt emoji/icon thường gặp trong bình luận
    \item Xử lý được teencode và ngôn ngữ không chuẩn ("k" = "không", "đc" = "được")
    \item Phù hợp với văn phong bình luận mua hàng hơn PhoBERT (huấn luyện trên báo chí)
\end{itemize}

\textbf{2. Chiến lược huấn luyện - Single-Task Learning (STL)}

Thay vì dùng 1 mô hình làm 2 việc cùng lúc (Multi-Task Learning), nhóm tách thành 2 mô hình riêng biệt:
\begin{itemize}
    \item \textbf{Mô hình AD}: Chuyên phát hiện khía cạnh (Aspect Detection)
    \item \textbf{Mô hình SC}: Chuyên phân loại cảm xúc (Sentiment Classification)
\end{itemize}

\textbf{Lý do:} STL giúp tránh task interference (xung đột nhiệm vụ), mỗi mô hình tập trung tối ưu cho một mục tiêu cụ thể.

\begin{figure}[H]
    \centering
    \begin{subfigure}[b]{0.48\textwidth}
        \centering
        \includegraphics[width=\textwidth]{image/sequence/ad_model.png}
        \caption{Mô hình AD (Aspect Detection)}
    \end{subfigure}
    \hfill
    \begin{subfigure}[b]{0.48\textwidth}
        \centering
        \includegraphics[width=\textwidth]{image/sequence/sc_model.png}
        \caption{Mô hình SC (Sentiment Classification)}
    \end{subfigure}
    \caption{Kiến trúc Single-Task Learning cho ABSA}
    \label{fig:stl_arch}
\end{figure}

\subsubsection*{Kỹ thuật Focal Loss - Xử lý mất cân bằng dữ liệu}

\textbf{Vấn đề:} Trong dữ liệu bình luận thực tế, số lượng đánh giá tích cực (Positive) thường áp đảo tiêu cực (Negative) và trung tính (Neutral). Tỷ lệ có thể lên đến 70-80\% Positive.

Hàm mất mát truyền thống Cross-Entropy khiến mô hình chỉ học tốt các mẫu "dễ" (lớp đa số) và bỏ qua các mẫu "khó" (lớp thiểu số).

\textbf{Giải pháp:} Áp dụng Focal Loss với tham số $\gamma=2$:

$$FL(p_{t})=-\alpha_{t}(1-p_{t})^{\gamma}\log(p_{t})$$

Trong đó:
\begin{itemize}
    \item $p_t$: Xác suất dự đoán đúng
    \item $\alpha_t$: Trọng số cân bằng lớp
    \item $\gamma$: Focusing parameter (thường = 2)
    \item $(1-p_t)^\gamma$: Modulating factor - giảm loss cho mẫu dễ, tăng cho mẫu khó
\end{itemize}

\textbf{Tác dụng:} Mô hình tập trung học các mẫu "khó" (Negative, Neutral), cải thiện đáng kể F1-Score cho các lớp thiểu số.


\subsubsection*{Tiền xử lý dữ liệu}

Trước khi đưa vào mô hình ABSA, dữ liệu được xử lý qua các bước:

\begin{table}[H]
\centering
\caption{Pipeline tiền xử lý dữ liệu}
\renewcommand{\arraystretch}{1.3}
\begin{tabular}{|l|l|l|}
\hline
\textbf{Bước} & \textbf{Input} & \textbf{Output} \\
\hline
Chuẩn hóa chính tả & "tooott", "đẹpppp" & "tốt", "đẹp" \\
\hline
Dịch tiếng Anh & "good", "ship fast" & "tốt", "giao nhanh" \\
\hline
Chuẩn hóa teencode & "k", "đc", "ko" & "không", "được", "không" \\
\hline
Giữ lại Emoji & (like), (love), (angry) & (giữ nguyên - mang sắc thái) \\
\hline
\end{tabular}
\end{table}

\subsubsection*{Kết quả thực nghiệm}

\textbf{Bộ dữ liệu:}
\begin{itemize}
    \item Số lượng: 14,912 bình luận từ các sàn TMĐT Việt Nam
    \item Độ đồng thuận gán nhãn: Kappa = 0.8208 (Rất tốt)
    \item Chia tỷ lệ: 80\% train, 10\% validation, 10\% test
\end{itemize}

\textbf{So sánh F1-Score các mô hình (STL vs MTL):}

\begin{table}[H]
\centering
\caption{So sánh F1-Score của các mô hình ABSA}
\renewcommand{\arraystretch}{1.25}
\begin{tabular}{|l|c|c|c|c|}
\hline
\multirow{2}{*}{\textbf{Mô hình}} & \multicolumn{2}{c|}{\textbf{STL}} & \multicolumn{2}{c|}{\textbf{MTL}} \\
\cline{2-5}
 & \textbf{AD F1} & \textbf{SC F1} & \textbf{AD F1} & \textbf{SC F1} \\
\hline
BiLSTM & 85.69\% & 39.83\% & 82.85\% & 34.28\% \\
\hline
BiLSTM + Conv1D & 86.23\% & 36.87\% & \textbf{84.09\%} & 33.48\% \\
\hline
PhoBERT & 88.84\% & \textbf{92.06\%} & 66.28\% & \textbf{92.93\%} \\
\hline
\textbf{VisoBERT} & \textbf{89.39\%} & \textbf{96.37\%} & 82.68\% & \textbf{93.63\%} \\
\hline
\end{tabular}
\end{table}

\subsubsection*{Phân tích kết quả}

\textbf{1. STL tốt hơn MTL:}
\begin{itemize}
    \item PhoBERT-STL: AD tăng +22.56\% so với MTL
    \item VisoBERT-STL: AD tăng +6.71\%, SC tăng +2.74\% so với MTL
\end{itemize}

\textbf{2. Transformer vượt trội BiLSTM:}
\begin{itemize}
    \item SC: 96.37\% (VisoBERT) vs 39.83\% (BiLSTM) = cải thiện +56.54\%
    \item Transformer có khả năng biểu diễn ngữ cảnh mạnh hơn
\end{itemize}

\textbf{3. VisoBERT tốt hơn PhoBERT:}
\begin{itemize}
    \item VisoBERT-STL: AD 89.39\%, SC 96.37\%
    \item PhoBERT-STL: AD 88.84\%, SC 92.06\%
    \item VisoBERT huấn luyện trên 14GB dữ liệu mạng xã hội, hiểu được emoji/icon
\end{itemize}

\subsection{Content Moderation}

Hệ thống tự động phát hiện và flag nội dung vi phạm trong đánh giá:

\textbf{Categories kiểm duyệt:} Harassment (quấy rối), Hate (thù ghét), Violence (bạo lực), Sexual (khiêu dâm), Self-harm (tự gây thương tích), Illicit (bất hợp pháp).

\textbf{Quy trình:} Review mới → Content Moderation API → Nếu phát hiện vi phạm: isFlagged=true, moderationStatus='pending' → Admin duyệt thủ công (Approve/Reject).

