\newpage
\section{\textbf{KẾT LUẬN VÀ HƯỚNG PHÁT TRIỂN}}

\subsection{Kết quả đạt được}

\begin{enumerate}
    \item \textbf{Xây dựng bộ dữ liệu}: Thu thập và đánh nhãn 14,912 bình luận từ các sàn thương mại điện tử Shopee, Lazada, Tiki với 11 khía cạnh và 3 mức cảm xúc. Độ đồng thuận Fleiss' Kappa đạt 0.8208 (mức rất tốt).
    
    \item \textbf{Thực nghiệm đa mô hình}: So sánh 6 mô hình thuộc 3 nhóm kiến trúc (BiLSTM, PhoBERT, VisoBERT) với 2 chiến lược huấn luyện (MTL và STL).
    
    \item \textbf{Kết quả tốt nhất}: Mô hình VisoBERT-STL đạt F1 = 89.39\% cho Phát hiện khía cạnh và F1 = 96.37\% cho Phân loại cảm xúc.
    
    \item \textbf{Tích hợp hệ thống}: Triển khai mô hình vào hệ thống thương mại điện tử thực tế với API phân tích cảm xúc theo thời gian thực.
\end{enumerate}

\subsection{Hạn chế cần khắc phục}

\begin{itemize}
    \item Bộ dữ liệu tập trung vào lĩnh vực điện thoại di động, cần mở rộng sang các lĩnh vực khác.
    \item Chưa xử lý được các bình luận có nhiều khía cạnh với cảm xúc trái ngược phức tạp.
    \item Thời gian suy luận của mô hình Transformer còn chậm so với yêu cầu thời gian thực.
\end{itemize}

\subsection{Hướng phát triển}

\begin{itemize}
    \item Mở rộng bộ dữ liệu sang các lĩnh vực khác như thời trang, mỹ phẩm, thực phẩm.
    \item Nghiên cứu kỹ thuật nén mô hình (Knowledge Distillation) để giảm thời gian suy luận.
    \item Tích hợp thêm phân tích hình ảnh sản phẩm để có đánh giá toàn diện hơn.
    \item Phát triển dashboard trực quan hóa xu hướng cảm xúc theo thời gian.
\end{itemize}

