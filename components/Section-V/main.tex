\newpage
\section{\textbf{PHÂN TÍCH VÀ THIẾT KẾ HỆ THỐNG}}

\subsection{Phân Tích Hệ Thống}

\textbf{Biểu đồ Use Case}

\textbf{Use Case Tổng quát} \\
Hệ thống "VeritaShop - Ứng Dụng Thương Mại Điện Tử Di Động" bao gồm các tác nhân chính và các chức năng tổng quát như sau:

\begin{itemize}
    \item \textbf{Khách hàng (Customer)}: Đăng ký, Đăng nhập, Xem sản phẩm, Tìm kiếm/Lọc, Thêm giỏ hàng, Quản lý Wishlist, Thanh toán (COD/MoMo), Xem lịch sử đơn hàng, Đánh giá sản phẩm.
    \item \textbf{Quản trị viên (Admin)}: Đăng nhập, Quản lý sản phẩm (CRUD), Quản lý đơn hàng, Quản lý người dùng, Quản lý mã giảm giá, Kiểm duyệt đánh giá, Xem thống kê Dashboard.
    \item \textbf{Hệ thống thanh toán (MoMo)}: Xử lý yêu cầu thanh toán, Gửi thông báo kết quả (IPN).
    \item \textbf{Hệ thống AI}: Phân tích cảm xúc đánh giá (ABSA), Kiểm duyệt nội dung (Content Moderation).
\end{itemize}

% Placeholder cho hình Use Case Tổng quát
\begin{figure}[H]
    \centering
    \includegraphics[width=1.0\textwidth]{image/sequence/2.png}
    \caption{Sơ đồ Use Case Tổng quát hệ thống VeritaShop}
    \label{fig:usecase_tongquat}
\end{figure}

\textbf{Đặc tả các Use Case chính}

\textbf{1. Xem và Tìm kiếm Sản phẩm} \\
Cho phép khách hàng duyệt danh sách sản phẩm với các bộ lọc và sắp xếp đa dạng.

\begin{longtable}{|>\bfseries m{4cm}|m{10cm}|}
\caption{Đặc tả Use Case: Xem và Tìm kiếm Sản phẩm}
\label{table:usecase-search}\\
\hline
Tên Use Case & Xem và Tìm kiếm Sản phẩm \\
\hline
Mô tả & Khách hàng duyệt danh sách điện thoại, tìm kiếm theo từ khóa và lọc theo tiêu chí. \\
\hline
Tác nhân & Khách hàng \\
\hline
Tiền điều kiện &
\begin{itemize}
    \item Ứng dụng đã được cài đặt và khởi động.
    \item Có kết nối mạng Internet.
\end{itemize} \\
\hline
Hậu điều kiện &
\begin{itemize}
    \item Danh sách sản phẩm phù hợp được hiển thị.
    \item Khách hàng có thể xem chi tiết sản phẩm.
\end{itemize} \\
\hline
Luồng sự kiện chính &
\begin{enumerate}
    \item Khách hàng mở màn hình danh sách sản phẩm.
    \item Hệ thống hiển thị danh sách sản phẩm với phân trang.
    \item Khách hàng nhập từ khóa tìm kiếm hoặc chọn bộ lọc.
    \item Bộ lọc gồm: Brand (iPhone, Samsung, Xiaomi, OPPO, Vivo), Giá (khoảng giá), Tình trạng (Mới, Like New, Đã dùng).
    \item Khách hàng chọn sắp xếp: Giá tăng/giảm, Rating cao nhất, Mới nhất.
    \item Hệ thống gọi API server-side search và trả về kết quả.
    \item Khách hàng chọn sản phẩm để xem chi tiết.
\end{enumerate} \\
\hline
Luồng thay thế &
\begin{itemize}
    \item \textbf{Không có kết quả}: Hiển thị thông báo "Không tìm thấy sản phẩm phù hợp".
    \item \textbf{Lỗi mạng}: Hiển thị thông báo và nút "Thử lại".
\end{itemize} \\
\hline
\end{longtable}

\textbf{2. Thanh toán Đơn hàng} \\
Quy trình hoàn tất đơn hàng với nhiều phương thức thanh toán.

\begin{longtable}{|>\bfseries m{4cm}|m{10cm}|}
\caption{Đặc tả Use Case: Thanh toán Đơn hàng}
\label{table:usecase-checkout}\\
\hline
Tên Use Case & Thanh toán Đơn hàng \\
\hline
Mô tả & Khách hàng hoàn tất mua hàng bằng COD hoặc MoMo. \\
\hline
Tác nhân & Khách hàng, Hệ thống MoMo \\
\hline
Tiền điều kiện &
\begin{itemize}
    \item Khách hàng đã đăng nhập.
    \item Giỏ hàng có ít nhất một sản phẩm.
    \item Có địa chỉ giao hàng.
\end{itemize} \\
\hline
Hậu điều kiện &
\begin{itemize}
    \item Đơn hàng được tạo với trạng thái "pending".
    \item Sản phẩm trong giỏ hàng bị xóa.
    \item Email xác nhận được gửi (nếu có).
\end{itemize} \\
\hline
Luồng sự kiện chính &
\begin{enumerate}
    \item Khách hàng chọn "Thanh toán" từ giỏ hàng.
    \item Hệ thống hiển thị màn hình Checkout với tóm tắt đơn hàng.
    \item Khách hàng chọn/thêm địa chỉ giao hàng.
    \item Khách hàng áp dụng mã giảm giá (coupon) nếu có.
    \item Khách hàng chọn phương thức thanh toán: COD hoặc MoMo.
    \item Nếu COD: Xác nhận đặt hàng, đơn được tạo ngay.
    \item Nếu MoMo: Chuyển hướng sang app MoMo, xác nhận thanh toán, đợi IPN callback.
    \item Hệ thống tạo đơn hàng và hiển thị màn hình "Đặt hàng thành công".
\end{enumerate} \\
\hline
Luồng thay thế &
\begin{itemize}
    \item \textbf{Sản phẩm hết hàng}: Thông báo và yêu cầu cập nhật giỏ hàng.
    \item \textbf{Mã giảm giá không hợp lệ}: Thông báo lỗi và cho phép tiếp tục không có coupon.
    \item \textbf{Thanh toán MoMo thất bại}: Quay về app, hiển thị "Giao dịch thất bại", cho phép thử lại.
\end{itemize} \\
\hline
\end{longtable}

\textbf{3. Đánh giá Sản phẩm (với ABSA)} \\
Khách hàng viết đánh giá và hệ thống tự động phân tích cảm xúc.

\begin{longtable}{|>\bfseries m{4cm}|m{10cm}|}
\caption{Đặc tả Use Case: Đánh giá Sản phẩm}
\label{table:usecase-review}\\
\hline
Tên Use Case & Đánh giá Sản phẩm với ABSA \\
\hline
Mô tả & Khách hàng đánh giá sản phẩm đã mua, hệ thống phân tích cảm xúc tự động. \\
\hline
Tác nhân & Khách hàng, Hệ thống AI (ABSA, Content Moderation) \\
\hline
Tiền điều kiện & Khách hàng đã đăng nhập và đã mua sản phẩm. \\
\hline
Hậu điều kiện & Đánh giá được lưu với kết quả phân tích cảm xúc. \\
\hline
Luồng sự kiện chính &
\begin{enumerate}
    \item Khách hàng chọn "Viết đánh giá" từ chi tiết sản phẩm hoặc lịch sử đơn hàng.
    \item Hệ thống hiển thị form đánh giá.
    \item Khách hàng chọn số sao (1-5), nhập tiêu đề và nội dung.
    \item Khách hàng upload hình ảnh (tùy chọn, tối đa 5 ảnh).
    \item Khách hàng gửi đánh giá.
    \item Backend thực hiện Content Moderation kiểm tra nội dung.
    \item Backend thực hiện ABSA phân tích cảm xúc theo aspects.
    \item Đánh giá được lưu với sentimentAnalysis và overallSentiment.
    \item Thông báo "Đánh giá thành công" (hoặc "Đang chờ duyệt" nếu bị flag).
\end{enumerate} \\
\hline
Luồng thay thế &
\begin{itemize}
    \item \textbf{Nội dung vi phạm}: Đánh giá bị flag, chuyển sang moderationStatus='pending' cho Admin duyệt.
\end{itemize} \\
\hline
\end{longtable}

\textbf{4. Quản lý Đơn hàng (Admin)} \\
Quản trị viên cập nhật trạng thái đơn hàng theo quy trình.

\begin{longtable}{|>\bfseries m{4cm}|m{10cm}|}
\caption{Đặc tả Use Case: Quản lý Đơn hàng}
\label{table:usecase-admin-order}\\
\hline
Tên Use Case & Quản lý Đơn hàng \\
\hline
Mô tả & Admin xem, cập nhật trạng thái và xử lý đơn hàng. \\
\hline
Tác nhân & Quản trị viên \\
\hline
Tiền điều kiện & Admin đã đăng nhập với role='admin'. \\
\hline
Hậu điều kiện & Trạng thái đơn hàng được cập nhật, timestamps được ghi nhận. \\
\hline
Luồng sự kiện chính &
\begin{enumerate}
    \item Admin truy cập màn hình Quản lý Đơn hàng.
    \item Hệ thống hiển thị danh sách đơn hàng với bộ lọc (trạng thái, ngày).
    \item Admin chọn đơn hàng để xem chi tiết.
    \item Admin cập nhật trạng thái theo quy trình:
    \begin{itemize}
        \item pending $\rightarrow$ confirmed $\rightarrow$ processing $\rightarrow$ shipping $\rightarrow$ delivered
        \item Hoặc: pending $\rightarrow$ cancelled
    \end{itemize}
    \item Hệ thống lưu trạng thái mới và timestamp tương ứng.
    \item Khách hàng nhận thông báo về trạng thái đơn hàng.
\end{enumerate} \\
\hline
\end{longtable}

\subsection{Thiết Kế Hệ Thống}

\textbf{Biểu đồ Activity (Hoạt động)}

\textbf{Quy trình Checkout và Thanh toán}

\begin{figure}[H]
    \centering
    \begin{subfigure}[b]{0.48\textwidth}
        \centering
        \includegraphics[width=0.7\textwidth]{image/sequence/3a.png}
        \caption{Giai đoạn chọn sản phẩm và địa chỉ}
    \end{subfigure}
    \hfill
    \begin{subfigure}[b]{0.48\textwidth}
        \centering
        \includegraphics[width=0.7\textwidth]{image/sequence/3b.png}
        \caption{Giai đoạn thanh toán và hoàn tất}
    \end{subfigure}
    \caption{Sơ đồ hoạt động chức năng Checkout}
    \label{fig:activity_checkout}
\end{figure}
\newpage
\textbf{Quy trình Thanh toán MoMo}

\begin{figure}[H]
    \centering
    \begin{subfigure}[b]{0.48\textwidth}
        \centering
        \includegraphics[width=0.7\textwidth]{image/sequence/4a.png}
        \caption{Giai đoạn tạo yêu cầu thanh toán}
    \end{subfigure}
    \hfill
    \begin{subfigure}[b]{0.48\textwidth}
        \centering
        \includegraphics[width=0.7\textwidth]{image/sequence/4b.png}
        \caption{Giai đoạn xử lý IPN callback}
    \end{subfigure}
    \caption{Sơ đồ hoạt động chức năng Thanh toán MoMo}
    \label{fig:activity_momo}
\end{figure}
\newpage
\textbf{Biểu đồ Sequence (Tuần tự)}

\textbf{Luồng Đánh giá Sản phẩm với ABSA}

% Placeholder cho hình Sequence Diagram
\begin{figure}[H]
    \centering
    \includegraphics[width=1.0\textwidth]{image/sequence/5.png}
    \caption{Sơ đồ tuần tự chức năng Đánh giá với ABSA}
    \label{fig:sequence_review}
\end{figure}

\textbf{Biểu đồ Luồng Màn hình (Screen Flow)}

\textbf{Luồng màn hình chính - Customer App}

\begin{enumerate}
    \item \textbf{Khởi động}: Splash Screen $\rightarrow$ Login/Register (nếu chưa đăng nhập) $\rightarrow$ Home
    \item \textbf{Mua sắm}: Home $\rightarrow$ Product List $\rightarrow$ Product Detail $\rightarrow$ Add to Cart hoặc Buy Now
    \item \textbf{Thanh toán}: Cart $\rightarrow$ Checkout $\rightarrow$ Address Selection $\rightarrow$ Payment $\rightarrow$ Payment Processing $\rightarrow$ Order Success
    \item \textbf{Đơn hàng}: Profile $\rightarrow$ Order History $\rightarrow$ Order Detail
    \item \textbf{Đánh giá}: Product Detail $\rightarrow$ Reviews $\rightarrow$ Write Review $\rightarrow$ ABSA Analysis Result
\end{enumerate}

